% Options for packages loaded elsewhere
\PassOptionsToPackage{unicode}{hyperref}
\PassOptionsToPackage{hyphens}{url}
%
\documentclass[
]{article}
\author{}
\date{}

\usepackage{amsmath,amssymb}
\usepackage{lmodern}
\usepackage{iftex}
\ifPDFTeX
  \usepackage[T1]{fontenc}
  \usepackage[utf8]{inputenc}
  \usepackage{textcomp} % provide euro and other symbols
\else % if luatex or xetex
  \usepackage{unicode-math}
  \defaultfontfeatures{Scale=MatchLowercase}
  \defaultfontfeatures[\rmfamily]{Ligatures=TeX,Scale=1}
\fi
% Use upquote if available, for straight quotes in verbatim environments
\IfFileExists{upquote.sty}{\usepackage{upquote}}{}
\IfFileExists{microtype.sty}{% use microtype if available
  \usepackage[]{microtype}
  \UseMicrotypeSet[protrusion]{basicmath} % disable protrusion for tt fonts
}{}
\makeatletter
\@ifundefined{KOMAClassName}{% if non-KOMA class
  \IfFileExists{parskip.sty}{%
    \usepackage{parskip}
  }{% else
    \setlength{\parindent}{0pt}
    \setlength{\parskip}{6pt plus 2pt minus 1pt}}
}{% if KOMA class
  \KOMAoptions{parskip=half}}
\makeatother
\usepackage{xcolor}
\IfFileExists{xurl.sty}{\usepackage{xurl}}{} % add URL line breaks if available
\IfFileExists{bookmark.sty}{\usepackage{bookmark}}{\usepackage{hyperref}}
\hypersetup{
  hidelinks,
  pdfcreator={LaTeX via pandoc}}
\urlstyle{same} % disable monospaced font for URLs
\usepackage[margin=2.5cm]{geometry}
\usepackage{color}
\usepackage{fancyvrb}
\newcommand{\VerbBar}{|}
\newcommand{\VERB}{\Verb[commandchars=\\\{\}]}
\DefineVerbatimEnvironment{Highlighting}{Verbatim}{commandchars=\\\{\}}
% Add ',fontsize=\small' for more characters per line
\newenvironment{Shaded}{}{}
\newcommand{\AlertTok}[1]{\textcolor[rgb]{1.00,0.00,0.00}{\textbf{#1}}}
\newcommand{\AnnotationTok}[1]{\textcolor[rgb]{0.38,0.63,0.69}{\textbf{\textit{#1}}}}
\newcommand{\AttributeTok}[1]{\textcolor[rgb]{0.49,0.56,0.16}{#1}}
\newcommand{\BaseNTok}[1]{\textcolor[rgb]{0.25,0.63,0.44}{#1}}
\newcommand{\BuiltInTok}[1]{#1}
\newcommand{\CharTok}[1]{\textcolor[rgb]{0.25,0.44,0.63}{#1}}
\newcommand{\CommentTok}[1]{\textcolor[rgb]{0.38,0.63,0.69}{\textit{#1}}}
\newcommand{\CommentVarTok}[1]{\textcolor[rgb]{0.38,0.63,0.69}{\textbf{\textit{#1}}}}
\newcommand{\ConstantTok}[1]{\textcolor[rgb]{0.53,0.00,0.00}{#1}}
\newcommand{\ControlFlowTok}[1]{\textcolor[rgb]{0.00,0.44,0.13}{\textbf{#1}}}
\newcommand{\DataTypeTok}[1]{\textcolor[rgb]{0.56,0.13,0.00}{#1}}
\newcommand{\DecValTok}[1]{\textcolor[rgb]{0.25,0.63,0.44}{#1}}
\newcommand{\DocumentationTok}[1]{\textcolor[rgb]{0.73,0.13,0.13}{\textit{#1}}}
\newcommand{\ErrorTok}[1]{\textcolor[rgb]{1.00,0.00,0.00}{\textbf{#1}}}
\newcommand{\ExtensionTok}[1]{#1}
\newcommand{\FloatTok}[1]{\textcolor[rgb]{0.25,0.63,0.44}{#1}}
\newcommand{\FunctionTok}[1]{\textcolor[rgb]{0.02,0.16,0.49}{#1}}
\newcommand{\ImportTok}[1]{#1}
\newcommand{\InformationTok}[1]{\textcolor[rgb]{0.38,0.63,0.69}{\textbf{\textit{#1}}}}
\newcommand{\KeywordTok}[1]{\textcolor[rgb]{0.00,0.44,0.13}{\textbf{#1}}}
\newcommand{\NormalTok}[1]{#1}
\newcommand{\OperatorTok}[1]{\textcolor[rgb]{0.40,0.40,0.40}{#1}}
\newcommand{\OtherTok}[1]{\textcolor[rgb]{0.00,0.44,0.13}{#1}}
\newcommand{\PreprocessorTok}[1]{\textcolor[rgb]{0.74,0.48,0.00}{#1}}
\newcommand{\RegionMarkerTok}[1]{#1}
\newcommand{\SpecialCharTok}[1]{\textcolor[rgb]{0.25,0.44,0.63}{#1}}
\newcommand{\SpecialStringTok}[1]{\textcolor[rgb]{0.73,0.40,0.53}{#1}}
\newcommand{\StringTok}[1]{\textcolor[rgb]{0.25,0.44,0.63}{#1}}
\newcommand{\VariableTok}[1]{\textcolor[rgb]{0.10,0.09,0.49}{#1}}
\newcommand{\VerbatimStringTok}[1]{\textcolor[rgb]{0.25,0.44,0.63}{#1}}
\newcommand{\WarningTok}[1]{\textcolor[rgb]{0.38,0.63,0.69}{\textbf{\textit{#1}}}}
\setlength{\emergencystretch}{3em} % prevent overfull lines
\providecommand{\tightlist}{%
  \setlength{\itemsep}{0pt}\setlength{\parskip}{0pt}}
\setcounter{secnumdepth}{-\maxdimen} % remove section numbering
\usepackage{fvextra}
\DefineVerbatimEnvironment{Highlighting}{Verbatim}{breaklines,commandchars=\\\{\}}
\ifLuaTeX
  \usepackage{selnolig}  % disable illegal ligatures
\fi

\begin{document}
\begin{center}
\hypertarget{st-thomas-college-thrissur}{%
\subsection{ST THOMAS COLLEGE ,
THRISSUR}\label{st-thomas-college-thrissur}}

\hypertarget{department-of-computer-science}{%
\subsection{Department of Computer
Science}\label{department-of-computer-science}}

\hypertarget{bsc-computer-science-2020-23---semester-iii}{%
\subsection{BSc Computer Science (2020-23) - Semester
III}\label{bsc-computer-science-2020-23---semester-iii}}
\end{center}

\hypertarget{assignment-2}{%
\subsection{Assignment 2}\label{assignment-2}}

\hypertarget{question}{%
\paragraph{Question}\label{question}}

The source code for a simple contact book program is given below. You
are asked to go through the code and its documentation. It lacks certain
features and you are expected to improve the functionalities listed
below.

\begin{enumerate}
\def\labelenumi{\arabic{enumi}.}
\tightlist
\item
  In the original , you can only store the contact's name and number.
  Add option to store contact's email and birthday.
\item
  The contacts are stored one after another. It would be better if they
  are added(inserted) alphabetically. (\emph{HINT}: strcmp())
\item
  The searching of a contact by name requires typing the exact name. It
  would be better if it could search with only a part of the name.
  (\emph{HINT}: naive search)
\item
  Add an option to search a contact by the contact's number.
\item
  The contact details are lost when the program exits. It would be good
  if the data can be stored in a file.
\end{enumerate}

\hypertarget{source-code}{%
\subparagraph{Source Code}\label{source-code}}

\begin{Shaded}
\begin{Highlighting}[]

\PreprocessorTok{\#include }\ImportTok{\textless{}stdio.h\textgreater{}}
\PreprocessorTok{\#include }\ImportTok{\textless{}string.h\textgreater{}}

\PreprocessorTok{\#define SIZE\_OF\_CONTACT\_LIST 100}

\CommentTok{// ********************************************}
\KeywordTok{struct}\NormalTok{ CONTACT }\OperatorTok{\{}

    \DataTypeTok{long} \DataTypeTok{int}\NormalTok{ number}\OperatorTok{;}
    \DataTypeTok{char}\NormalTok{ name}\OperatorTok{[}\DecValTok{50}\OperatorTok{];}
\OperatorTok{\};}

\KeywordTok{typedef} \KeywordTok{struct}\NormalTok{ CONTACT CONTACT}\OperatorTok{;}

\NormalTok{CONTACT List}\OperatorTok{[}\NormalTok{SIZE\_OF\_CONTACT\_LIST}\OperatorTok{];}
\DataTypeTok{int}\NormalTok{ last\_pos}\OperatorTok{=}\DecValTok{0}\OperatorTok{;}

\DataTypeTok{void}\NormalTok{ add\_contact}\OperatorTok{(}\NormalTok{CONTACT c}\OperatorTok{)\{}
\NormalTok{    List}\OperatorTok{[}\NormalTok{last\_pos}\OperatorTok{]=}\NormalTok{c}\OperatorTok{;}
\NormalTok{    last\_pos}\OperatorTok{++;}
\OperatorTok{\}}

\DataTypeTok{void}\NormalTok{ delete\_contact}\OperatorTok{(}\DataTypeTok{int}\NormalTok{ contact\_index}\OperatorTok{)\{}

\NormalTok{    CONTACT temp }\OperatorTok{=}\NormalTok{ List}\OperatorTok{[}\NormalTok{contact\_index}\OperatorTok{];}

    \ControlFlowTok{for}\OperatorTok{(}\DataTypeTok{int}\NormalTok{ i}\OperatorTok{=}\NormalTok{contact\_index}\OperatorTok{;}\NormalTok{i}\OperatorTok{\textless{}}\NormalTok{last\_pos}\OperatorTok{;}\NormalTok{i}\OperatorTok{++)}
\NormalTok{        List}\OperatorTok{[}\NormalTok{i}\OperatorTok{]=}\NormalTok{List}\OperatorTok{[}\NormalTok{i}\OperatorTok{+}\DecValTok{1}\OperatorTok{];}

\NormalTok{    last\_pos}\OperatorTok{{-}{-};}
\OperatorTok{\}}

\DataTypeTok{void}\NormalTok{ edit\_contact}\OperatorTok{(}\NormalTok{CONTACT edited\_contact}\OperatorTok{,}\DataTypeTok{int}\NormalTok{ index}\OperatorTok{)\{}

\NormalTok{    List}\OperatorTok{[}\NormalTok{index}\OperatorTok{]=}\NormalTok{edited\_contact}\OperatorTok{;}

\OperatorTok{\}}

\DataTypeTok{void}\NormalTok{ display\_contact}\OperatorTok{(}\NormalTok{CONTACT C}\OperatorTok{)\{}

\NormalTok{    printf}\OperatorTok{(}\StringTok{"}\SpecialCharTok{\textbackslash{}n}\StringTok{{-}{-}{-}{-}{-}{-}{-}{-}{-}{-}{-}{-}{-}{-}{-}{-}{-}{-}{-}{-}{-}{-}{-}{-}{-}{-}{-}{-}{-}{-}{-}{-}{-}{-}{-}{-}{-}{-}{-}{-}{-}{-}{-}{-}{-}{-}{-}{-}{-}{-}{-}{-}{-}}\SpecialCharTok{\textbackslash{}n}\StringTok{"}\OperatorTok{);}
\NormalTok{    printf}\OperatorTok{(}\StringTok{"}\SpecialCharTok{\textbackslash{}t}\StringTok{Name         : \%s}\SpecialCharTok{\textbackslash{}n}\StringTok{"}\OperatorTok{,}\NormalTok{C}\OperatorTok{.}\NormalTok{name}\OperatorTok{);}
\NormalTok{    printf}\OperatorTok{(}\StringTok{"}\SpecialCharTok{\textbackslash{}t}\StringTok{Phone Number : \%li}\SpecialCharTok{\textbackslash{}n}\StringTok{"}\OperatorTok{,}\NormalTok{C}\OperatorTok{.}\NormalTok{number}\OperatorTok{);}
\NormalTok{    printf}\OperatorTok{(}\StringTok{"{-}{-}{-}{-}{-}{-}{-}{-}{-}{-}{-}{-}{-}{-}{-}{-}{-}{-}{-}{-}{-}{-}{-}{-}{-}{-}{-}{-}{-}{-}{-}{-}{-}{-}{-}{-}{-}{-}{-}{-}{-}{-}{-}{-}{-}{-}{-}{-}{-}{-}{-}{-}{-}}\SpecialCharTok{\textbackslash{}n}\StringTok{"}\OperatorTok{);}

\OperatorTok{\}}


\DataTypeTok{int}\NormalTok{ search\_contact\_by\_name}\OperatorTok{(}\DataTypeTok{char}\NormalTok{ name}\OperatorTok{[])\{}

    \ControlFlowTok{for}\OperatorTok{(}\DataTypeTok{int}\NormalTok{ i}\OperatorTok{=}\DecValTok{0}\OperatorTok{;}\NormalTok{i}\OperatorTok{\textless{}}\NormalTok{last\_pos}\OperatorTok{;}\NormalTok{i}\OperatorTok{++)\{}
        \ControlFlowTok{if}\OperatorTok{(}\NormalTok{ strcmp}\OperatorTok{(}\NormalTok{name}\OperatorTok{,}\NormalTok{List}\OperatorTok{[}\NormalTok{i}\OperatorTok{].}\NormalTok{name}\OperatorTok{)} \OperatorTok{==} \DecValTok{0}\OperatorTok{)}
            \ControlFlowTok{return}\NormalTok{ i}\OperatorTok{;}
    \OperatorTok{\}}

    \ControlFlowTok{return} \OperatorTok{{-}}\DecValTok{1}\OperatorTok{;}
\OperatorTok{\}}

\DataTypeTok{void}\NormalTok{ accept\_contact\_details}\OperatorTok{(}\NormalTok{CONTACT }\OperatorTok{*}\NormalTok{c}\OperatorTok{)\{}
\NormalTok{    printf}\OperatorTok{(}\StringTok{"}\SpecialCharTok{\textbackslash{}n}\StringTok{Enter name : "}\OperatorTok{);}
\NormalTok{    scanf}\OperatorTok{(}\StringTok{"\%s"}\OperatorTok{,(}\NormalTok{c}\OperatorTok{{-}\textgreater{}}\NormalTok{name}\OperatorTok{));}
\NormalTok{    printf}\OperatorTok{(}\StringTok{"Enter phone number : "}\OperatorTok{);}
\NormalTok{    scanf}\OperatorTok{(}\StringTok{"\%li"}\OperatorTok{,\&(}\NormalTok{c}\OperatorTok{{-}\textgreater{}}\NormalTok{number}\OperatorTok{));}
\OperatorTok{\}}

\CommentTok{//**************************************************************}

\DataTypeTok{void}\NormalTok{ display\_contacts}\OperatorTok{(}\NormalTok{CONTACT list}\OperatorTok{[],}\DataTypeTok{int}\NormalTok{ length}\OperatorTok{)\{}
\NormalTok{    printf}\OperatorTok{(}\StringTok{"}\SpecialCharTok{\textbackslash{}n}\StringTok{Contact List}\SpecialCharTok{\textbackslash{}n\textbackslash{}n}\StringTok{"}\OperatorTok{);}
    \ControlFlowTok{for}\OperatorTok{(}\DataTypeTok{int}\NormalTok{ i}\OperatorTok{=}\DecValTok{0}\OperatorTok{;}\NormalTok{i}\OperatorTok{\textless{}}\NormalTok{length}\OperatorTok{;}\NormalTok{i}\OperatorTok{++)\{}
\NormalTok{        printf}\OperatorTok{(}\StringTok{"\%d )"}\OperatorTok{,}\NormalTok{i}\OperatorTok{+}\DecValTok{1}\OperatorTok{);}
\NormalTok{        display\_contact}\OperatorTok{(}\NormalTok{list}\OperatorTok{[}\NormalTok{i}\OperatorTok{]);}
\NormalTok{        printf}\OperatorTok{(}\StringTok{"}\SpecialCharTok{\textbackslash{}n}\StringTok{"}\OperatorTok{);}
    \OperatorTok{\}}
\OperatorTok{\}}

\CommentTok{//***************************************************************}

\DataTypeTok{void}\NormalTok{ menu\_display}\OperatorTok{()\{}

\NormalTok{    printf}\OperatorTok{(}\StringTok{"}\SpecialCharTok{\textbackslash{}n}\StringTok{1. Add Contact"}\OperatorTok{);}
\NormalTok{    printf}\OperatorTok{(}\StringTok{"}\SpecialCharTok{\textbackslash{}n}\StringTok{2. Delete Contact"}\OperatorTok{);}
\NormalTok{    printf}\OperatorTok{(}\StringTok{"}\SpecialCharTok{\textbackslash{}n}\StringTok{3. Edit Contact"}\OperatorTok{);}
\NormalTok{    printf}\OperatorTok{(}\StringTok{"}\SpecialCharTok{\textbackslash{}n}\StringTok{4. Display Contacts"}\OperatorTok{);}
\NormalTok{    printf}\OperatorTok{(}\StringTok{"}\SpecialCharTok{\textbackslash{}n}\StringTok{5. Search Contact By Name"}\OperatorTok{);}
\NormalTok{    printf}\OperatorTok{(}\StringTok{"}\SpecialCharTok{\textbackslash{}n}\StringTok{0. Exit}\SpecialCharTok{\textbackslash{}n}\StringTok{"}\OperatorTok{);}
\NormalTok{    printf}\OperatorTok{(}\StringTok{"Enter option : "}\OperatorTok{);}

\OperatorTok{\}}

\DataTypeTok{void}\NormalTok{ menu}\OperatorTok{()\{}
    \DataTypeTok{int}\NormalTok{ choice}\OperatorTok{=}\DecValTok{1}\OperatorTok{;}

    \ControlFlowTok{while}\OperatorTok{(}\NormalTok{choice}\OperatorTok{)\{}

\NormalTok{        menu\_display}\OperatorTok{();}
\NormalTok{        scanf}\OperatorTok{(}\StringTok{"\%d"}\OperatorTok{,\&}\NormalTok{choice}\OperatorTok{);}

        \ControlFlowTok{switch}\OperatorTok{(}\NormalTok{choice}\OperatorTok{)\{}

            \ControlFlowTok{case} \DecValTok{1}\OperatorTok{:} \OperatorTok{\{}
\NormalTok{                        CONTACT C}\OperatorTok{;}
\NormalTok{                        accept\_contact\_details}\OperatorTok{(\&}\NormalTok{C}\OperatorTok{);}
\NormalTok{                        add\_contact}\OperatorTok{(}\NormalTok{C}\OperatorTok{);}
                    \OperatorTok{\}}
                    \ControlFlowTok{break}\OperatorTok{;}

            \ControlFlowTok{case} \DecValTok{2}\OperatorTok{:} \OperatorTok{\{}
                        \DataTypeTok{int}\NormalTok{ index}\OperatorTok{;}
\NormalTok{                        printf}\OperatorTok{(}\StringTok{"Enter Index : "}\OperatorTok{);}
\NormalTok{                        scanf}\OperatorTok{(}\StringTok{"\%d"}\OperatorTok{,\&}\NormalTok{index}\OperatorTok{);}
\NormalTok{                        delete\_contact}\OperatorTok{(}\NormalTok{index}\OperatorTok{{-}}\DecValTok{1}\OperatorTok{);}
                    \OperatorTok{\}}
                    \ControlFlowTok{break}\OperatorTok{;}

            \ControlFlowTok{case} \DecValTok{3}\OperatorTok{:} \OperatorTok{\{}
                        \DataTypeTok{int}\NormalTok{ index}\OperatorTok{;}

\NormalTok{                        printf}\OperatorTok{(}\StringTok{"Enter Index : "}\OperatorTok{);}
\NormalTok{                        scanf}\OperatorTok{(}\StringTok{"\%d"}\OperatorTok{,\&}\NormalTok{index}\OperatorTok{);}
\NormalTok{                        CONTACT C}\OperatorTok{;}
\NormalTok{                        accept\_contact\_details}\OperatorTok{(\&}\NormalTok{C}\OperatorTok{);}

                        \CommentTok{// display original contact}
\NormalTok{                        printf}\OperatorTok{(}\StringTok{"}\SpecialCharTok{\textbackslash{}n}\StringTok{Original Contact : "}\OperatorTok{);}
\NormalTok{                        display\_contact}\OperatorTok{(}\NormalTok{List}\OperatorTok{[}\NormalTok{index}\OperatorTok{{-}}\DecValTok{1}\OperatorTok{]);}

                        \CommentTok{// display edited contact}
\NormalTok{                        printf}\OperatorTok{(}\StringTok{"}\SpecialCharTok{\textbackslash{}n}\StringTok{Modified Contact : "}\OperatorTok{);}
\NormalTok{                        display\_contact}\OperatorTok{(}\NormalTok{C}\OperatorTok{);}

                        \DataTypeTok{char}\NormalTok{ c}\OperatorTok{;}
\NormalTok{                        printf}\OperatorTok{(}\StringTok{"Save edited contact?(Y/N)? "}\OperatorTok{);}
\NormalTok{                        scanf}\OperatorTok{(}\StringTok{" \%c"}\OperatorTok{,\&}\NormalTok{c}\OperatorTok{);}
                        \ControlFlowTok{if}\OperatorTok{(}\NormalTok{c}\OperatorTok{==}\CharTok{\textquotesingle{}y\textquotesingle{}}\OperatorTok{||}\NormalTok{c}\OperatorTok{==}\CharTok{\textquotesingle{}Y\textquotesingle{}}\OperatorTok{)}
\NormalTok{                            edit\_contact}\OperatorTok{(}\NormalTok{C}\OperatorTok{,}\NormalTok{index}\OperatorTok{{-}}\DecValTok{1}\OperatorTok{);}
                    \OperatorTok{\}}
                    \ControlFlowTok{break}\OperatorTok{;}

            \ControlFlowTok{case} \DecValTok{4}\OperatorTok{:}\NormalTok{ display\_contacts}\OperatorTok{(}\NormalTok{List}\OperatorTok{,}\NormalTok{last\_pos}\OperatorTok{);}
                    \ControlFlowTok{break}\OperatorTok{;}

            \ControlFlowTok{case} \DecValTok{5}\OperatorTok{:} \OperatorTok{\{}
                        \DataTypeTok{char}\NormalTok{ name}\OperatorTok{[}\DecValTok{50}\OperatorTok{];}
\NormalTok{                        printf}\OperatorTok{(}\StringTok{"Enter Name : "}\OperatorTok{);}
\NormalTok{                        scanf}\OperatorTok{(}\StringTok{"\%s"}\OperatorTok{,}\NormalTok{name}\OperatorTok{);}

                        \DataTypeTok{int}\NormalTok{ index}\OperatorTok{=}\NormalTok{search\_contact\_by\_name}\OperatorTok{(}\NormalTok{name}\OperatorTok{);}
                        \ControlFlowTok{if}\OperatorTok{(}\NormalTok{index }\OperatorTok{==} \OperatorTok{{-}}\DecValTok{1}\OperatorTok{)\{}
\NormalTok{                            printf}\OperatorTok{(}\StringTok{"Not Found"}\OperatorTok{);}
                            \ControlFlowTok{break}\OperatorTok{;}
                        \OperatorTok{\}}
\NormalTok{                        printf}\OperatorTok{(}\StringTok{"\%d )"}\OperatorTok{,}\NormalTok{index}\OperatorTok{+}\DecValTok{1}\OperatorTok{);}
\NormalTok{                        display\_contact}\OperatorTok{(}\NormalTok{List}\OperatorTok{[}\NormalTok{index}\OperatorTok{]);}

                    \OperatorTok{\}} 
                    \ControlFlowTok{break}\OperatorTok{;}
        \OperatorTok{\}}
    \OperatorTok{\}}

\OperatorTok{\}}

\DataTypeTok{int}\NormalTok{ main}\OperatorTok{(}\DataTypeTok{void}\OperatorTok{)} \OperatorTok{\{}

\NormalTok{    menu}\OperatorTok{();}

    \ControlFlowTok{return} \DecValTok{0}\OperatorTok{;}
\OperatorTok{\}}
\end{Highlighting}
\end{Shaded}

\hypertarget{documentation-for-simple-contacts-code}{%
\paragraph{Documentation}\label{documentation-for-simple-contacts-code}}

\begin{itemize}
\tightlist
\item
  The contact details is stored in a structure name \texttt{CONTACT}.
\item
  The contacts list is stored as a global array named \texttt{List}.
\item
  New contacts are appended to the contact lists and \texttt{last\_pos}
  stores the index of the last contact stored(i.e.~length of list)
\item
  \texttt{add\_contact()} is the function that appends the contact to
  the contacts list and updates \texttt{last\_pos}

  \begin{itemize}
  \tightlist
  \item
    argument : a \texttt{CONTACT} object that is to be added
  \end{itemize}
\item
  \texttt{delete\_contact()} deletes a contact at the index passed to it
  , shifts the contacts to the left and updates \texttt{last\_pos}

  \begin{itemize}
  \tightlist
  \item
    argument : the index of the contact to be deleted
  \end{itemize}
\item
  \texttt{edit\_contact()} replace a contact with an edited copy of
  itself.

  \begin{itemize}
  \tightlist
  \item
    arguments : an \texttt{CONTACT} object with the modified details ,
    index of the original contact
  \end{itemize}
\item
  \texttt{accept\_contact\_details()} : accepts data about the contact
  from the user and stores it a contact

  \begin{itemize}
  \tightlist
  \item
    argument : address of the \texttt{CONTACT} object where the data
    from the user needs to be stored
  \end{itemize}
\item
  \texttt{display\_contact()} : display the details of a contact with
  pretty formatting

  \begin{itemize}
  \tightlist
  \item
    argument : a \texttt{CONTACT} object
  \end{itemize}
\item
  \texttt{search\_contact\_by\_name()} : searches the contacts list for
  an exact match (case sensitive).

  \begin{itemize}
  \tightlist
  \item
    argument : a string name that needs to searched
  \item
    return value : a whole number,i.e., index of the matched contact or
    -1 meaning no matches
  \end{itemize}
\end{itemize}

\end{document}
